% $Log: abstract.tex,v $
% Revision 1.1  93/05/14  14:56:25  starflt
% Initial revision
% 
% Revision 1.1  90/05/04  10:41:01  lwvanels
% Initial revision
% 
%
%% The text of your abstract and nothing else (other than comments) goes here.
%% It will be single-spaced and the rest of the text that is supposed to go on
%% the abstract page will be generated by the abstractpage environment.  This
%% file should be \input (not \include 'd) from cover.tex.
\chapter*{Resumen}
\addcontentsline{toc}{chapter}{Resumen} % si queremos que aparezca en el índice

En esta tesis desarrollaremos una herramienta de verificación de modelos (Model Checking), llamada MC2, sobre programas formalizados en un lenguaje lógico simple que también será desarrollado en este trabajo. La herramienta de verificación se encargará de, valga la redundancia, verificar propiedades, caracterizadas en la forma de la lógica temporal Calculo Mu, sobre dichos programa. Cabe destacar que el lenguaje de programación y el verificador funcionan como una sola unidad ya que en un mismo programa MC2 se define tanto el modelo como las propiedades a verificar (si las hay). A todo esto, el metalenguaje utilizado para el desarrollo de estas herramientas es Haskell.

Una motivación para el desarrollo de esta tesis fue la utilización de un lenguaje funcional (Haskell) para el desarrollo de la herramienta en su totalidad, en lugar de utilizar lenguajes imperativos u orientados a objetos.

Lo que se propone con este proyecto es explorar otras alternativas para especificar propiedades que un modelo deba satisfacer, siendo la alternativa en este trabajo el Cálculo Mu, en vez de las lógicas temporales más utilizadas en este tipo de herramientas, como ser LTL, CTL y CTL*, ya que esto trae la ventaja de que el Cálculo Mu es más expresivo que los anteriormente nombrados. Adicionalmente los modelos de los programas MC2 se basan en la definición de reglas de transición usando proposiciones lógicas atómicas con lo cuál es transparente ver la estructura del modelo como una máquina de transición de estados más allá de que internamente se los trata simbólicamente como fórmulas para mejorar el rendimiento.


\emph{Palabras clave}: .

\thispagestyle{empty} % para que no se numere esta página
%\end{abstract}

\chapter*{Agradecimientos}
\addcontentsline{toc}{chapter}{Agradecimientos} % si queremos que aparezca en el índice

{\sl Agradezco a ...}\\

{\sl Una dedicatoria muy especial es para ...}\\

