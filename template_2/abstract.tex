% $Log: abstract.tex,v $
% Revision 1.1  93/05/14  14:56:25  starflt
% Initial revision
% 
% Revision 1.1  90/05/04  10:41:01  lwvanels
% Initial revision
% 
%
%% The text of your abstract and nothing else (other than comments) goes here.
%% It will be single-spaced and the rest of the text that is supposed to go on
%% the abstract page will be generated by the abstractpage environment.  This
%% file should be \input (not \include 'd) from cover.tex.
\chapter*{Resumen}
\pagenumbering{roman}
\addcontentsline{toc}{chapter}{Resumen} % si queremos que aparezca en el índice

En esta tesis desarrollaremos una herramienta de verificación de modelos (\emph Model \emph Checking), llamada MC2, sobre modelos de sistemas formalizados en un lenguaje de descripción simple que también será desarrollado en este trabajo. La herramienta de verificación se encargará de, valga la redundancia, verificar propiedades, caracterizadas en la forma de la lógica temporal Cálculo-$\mu$ (o cálculo-Mu), sobre algun modelo descripto. Cabe destacar que tanto el lenguaje de descripción de modelos como el verificador forman parte de la misma herramienta, incluso las propiedades a verificar se deben especificar en la descripción.

Adicionalmente, los descripciones MC2 se basan en la definición de reglas de transición usando proposiciones lógicas atómicas con lo cuál es transparente ver la estructura del modelo como una máquina de transición de estados más allá de que internamente se los trata simbólicamente como fórmulas para mejorar el rendimiento.

A todo esto, el metalenguaje utilizado para el desarrollo de estas herramientas es Haskell. Esto también significó una motivación para el desarrollo de esta tesis ya que la utilización de un lenguaje funcional como Haskell para el desarrollo de la herramienta en su totalidad, en lugar de la utilización lenguajes imperativos u orientados a objetos, da un enfoque distinto al comportamiento de la herramienta.

Lo que se propone con este proyecto es explorar otras alternativas para especificar propiedades que un modelo deba satisfacer, siendo la alternativa en este trabajo el Cálculo Mu, en vez de las lógicas temporales más utilizadas en este tipo de herramientas, como ser LTL, CTL y CTL*, ya que esto trae la ventaja de que el Cálculo Mu es más expresivo que los anteriormente nombrados. MC2 puede servir como herramienta de bajo nivel, en el sentido de que se puede extraer una descripción MC2 a partir de modelos de mas alto nivel, y asi mismo, propiedades en lógicas temporales de más alto nivel, para asi despues usar el verificador MC2 sobre la descripción extraida.


%\emph{Palabras clave}: .

%\thispagestyle{empty} % para que no se numere esta página
%\end{abstract}

\chapter*{Agradecimientos}
\addcontentsline{toc}{chapter}{Agradecimientos} % si queremos que aparezca en el índice
%\thispagestyle{empty} 
{\sl Agradezco a la universidad, el departamento de computación, y en especial a mi familia y mis amigos, esto no seria posible sin ellos.}

