%% This is an example first chapter.  You should put chapter/appendix that you
%% write into a separate file, and add a line \include{yourfilename} to
%% main.tex, where `yourfilename.tex' is the name of the chapter/appendix file.
%% You can process specific files by typing their names in at the 
%% \files=
%% prompt when you run the file main.tex through LaTeX.
\chapter{Introducción}
\pagenumbering{arabic}
La verificación de modelos o comunmente \emph{model checking} es una técnica automática para verificar sistemas reactivos con una cantidad finita de estados, por ejemplo protocolos de comunicación y diseños de circuitos. Las especificaciones de las propiedades a verificar son expresadas en una lógica temporal proposicional, y el sistema esta modelado como un grafo. Se utiliza una busqueda eficiente para determinar automáticamente si las especificaciones son satisfechas por el grafo \cite{Clarke:5}. Esta técnica fue desarrollada originalmente en 1981 por Clarke y Emerson. Quielle y Sifakis descubrieron independientemente una técnica similar de verificación poco después.
Esta técnica tiene varias ventajas importantes sobre probadores de teoremas para verificación de circuitos y protocolos. La mas importante es que es automática. Normalmente, el usuario provee una representación de alto nivel del modelo y una especificación de la propiedad que se desea verificar. El model checker terminará devolviendo la respuesta True indicando que el modelo satisface la especificación o dará una traza de ejecución a modo de contraejemplo si el modelo no satisface la propiedad. Esta es una propiedad muy importante a la hora de encontrar bugs sutiles.

\section{Verificación de modelos}

\section{Objetivos}
El objetivo principal de este proyecto es explorar otras alternativas para especificar propiedades que un modelo deba satisfacer, siendo la alternativa en este trabajo el Cálculo-$\mu$, en lugar de las lógicas temporales más utilizadas en este tipo de herramientas, como ser LTL, CTL y CTL*, además de que esto trae la ventaja de que el Cálculo-$\mu$ es más expresivo que los anteriormente nombrados, por lo tanto, las propiedades descritas en LTL,CTL y CTL* pueden ser descritas también usando Cálculo-$\mu$.
Como objetivo secundario cabe destacar la utilización del paradigma funcional de programación para el desarrollo de la herramienta en su totalidad, en lugar de utilizar paradigmas imperativos u orientados a objetos que son normalmente mas utilizados en el área. En cuanto a la aplicación práctica de la herramienta, la misma esta planeada para verificar propiedades en lógicas donde el problema de la verificación de modelos pueda ser reducida a cálculo-$mu$, por ejemplo dCTL \cite{Castro:9}, una lógica temporal deóntica usada para especificar propiedades sobre sistemas tolerantes a fallos.

\section{Estructura}
Primero analizaremos conceptos básicos para la comprensión de esta tesis, conceptos como la verificación de modelos, representación de estos modelos, el concepto de lógicas temporales, y en particular el Cálculo-$\mu$.
Más tarde introduciremos la noción de verificación simbólica de modelos para así luego entrar en detalle sobre la implementación del verificador de modelos MC2.
Luego estableceremos la idea detrás del lenguaje MC2, para entrar luego en detalle con la sintaxis y la semántica del lenguaje. Para terminar, se analizarán detalles concretos sobre la implementación de la herramienta. 
Por último veremos algunos ejemplos para afianzar el entendimiento de la aplicación práctica de esta herramienta.

\section{Desarrollo}
Primero se desarrollo una versión del verificador con estados explicitos. Luego se quiso hacer el verificador simbolico pero para un lenguaje mas complejo (con sentencias, estructuras de control, etc.), y despues de notar que se escapaba de la idea principal de la tesis y de mi tiempo, se encontró el punto medio que era hacer un verificador para modelos simbolicos en un lenguaje simple.


