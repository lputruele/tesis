%% This is an example first chapter.  You should put chapter/appendix that you
%% write into a separate file, and add a line \include{yourfilename} to
%% main.tex, where `yourfilename.tex' is the name of the chapter/appendix file.
%% You can process specific files by typing their names in at the 
%% \files=
%% prompt when you run the file main.tex through LaTeX.
\chapter{Introducción}
\pagenumbering{arabic}
\noindent La verificación de modelos, o \emph{model checking}, es una técnica automática para verificar modelos de sistemas con una cantidad finita de estados. Por ejemplo, protocolos de comunicación y diseños de circuitos, entre otros. En general, la aplicación de esta técnica consta de una caracterización del modelo en términos de algún tipo de sistema de transición de estados y la especificación de la propiedad a analizar en algún tipo de lenguaje formal, comúnmente algún tipo de lógica temporal.\\
\\
\noindent La idea principal es utilizar búsquedas eficientes sobre el grafo que describe el sistema a verificar, para determinar automáticamente si las especificaciones son satisfechas por el mismo \cite{Clarke:5}. Esta técnica fue desarrollada originalmente en 1981 por Clarke y Emerson. Se puede destacar que el model checking posee varias ventajas importantes sobre probadores de teoremas para verificación de circuitos y protocolos. La más importante es que es automática. Normalmente, el usuario provee una representación de alto nivel del modelo y una especificación de la propiedad que se desea verificar. El model checker terminará devolviendo la respuesta True indicando que el modelo satisface la especificación o dará una traza de ejecución a modo de contraejemplo si el modelo no satisface la propiedad. Los contraejemplos permiten hacer \emph{debugging} en el código fuente y encontrar partes defectuosas en los programas, por lo cual resulta ser una propiedad muy importante a la hora de encontrar bugs sutiles.

\section{Objetivos}
El objetivo principal de este proyecto es proveer una herramienta de verificación de modelos puramente funcional además de explorar otras alternativas para especificar propiedades que un modelo deba satisfacer, siendo la alternativa en este trabajo el {\mucalculo}, un lenguaje altamente expresivo, lo cuál trae la ventaja de que propiedades descritas en muchos otros tipos de lógicas temporales pueden ser descritos también por el {\mucalculo}. En cuanto a la aplicación práctica de la herramienta, la misma esta planeada para verificar propiedades en lógicas donde el problema de la verificación de modelos pueda ser reducida a {\mucalculo}, por ejemplo dCTL \cite{Castro:9}, una lógica temporal deóntica usada para especificar propiedades sobre sistemas tolerantes a fallas.

\section{Estructura}
En el capítulo 2 analizaremos conceptos básicos para la comprensión de esta tesis, conceptos como la verificación de modelos, representación de estos modelos, el concepto de lógicas temporales, y en particular, el {\mucalculo}.
En el capítulo 3 veremos formas de representar los datos de una manera más concisa e introduciremos la noción de verificación simbólica de modelos.
En el capítulo 4 nos adentramos en la herramienta MC2, su lenguaje de descripción de modelos y su verificador de modelos. Se entra en detalle con la sintaxis y la semántica del lenguaje. Para terminar, se analizarán cuestiones de diseño e implementación de la herramienta. 
En el capítulo 5 aplicaremos la herramienta a algunos casos de estudio concretos a modo de ejemplificación del uso práctico de esta herramienta.

\section{Repositorio de la herramienta}

La herramienta MC2 es \emph{open source} y completamente programada en el lenguaje funcional Haskell \cite{Haskell:18}. El código fuente y los ejemplos utilizados en esta tesis se pueden encontrar en el mismo:\\
\\
https://github.com/lputruele/MC2-Mu-Calculus-Model-Checker



