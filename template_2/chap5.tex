\chapter{Casos de estudio}

\section{Cruce del rio}

Once upon a time a farmer went to a market and purchased a fox, a goose, and a bag of beans. On his way home, the farmer came to the bank of a river and rented a boat. But in crossing the river by boat, the farmer could carry only himself and a single one of his purchases—the fox, the goose, or the bag of beans.

If left together, the fox would eat the goose, or the goose would eat the beans.

The farmer's challenge was to carry himself and his purchases to the far bank of the river, leaving each purchase intact. How did he do it? \cite{Hadley:12}

\section{1,2,3, Coloca otra vez}

Se trata de un juego de tipo solitario.
Se parte de una estrella de 5 puntas.
Partiendo de uno de los huecos en los que no haya una ficha previamente colocada, contaremos tres posiciones consecutivas sobre una de las aristas que contienen la posición de partida. Tras ello, colocaremos una ficha en la tercera posición (la última del conteo).
El conteo puede pasar por una posición en la que haya ficha, pero no puede iniciarse en una posición con ficha.
El juego estará resuelto cuando se hayan colocado 9 fichas. \cite{Juegos:11}

\section{Ranas saltarinas}

Se trata de un juego de tipo solitario. Para un sólo jugador.
Se parte de una tira de papel dividida en siete casillas.
La posición inicial es la indicada con tres fichas azules y tres rojas colocadas como en la figura de abajo.
El objetivo del juego consiste en permutar las posiciones de las fichas azules y rojas. Es decir, las azules han de pasar a ocupar las posiciones de las rojas y viceversa. Para ello son válidos los siguientes movimientos:
a) Una ficha puede moverse a un lugar contiguo, si éste está vacío.
b) Una ficha junto a otra de distinto color puede saltar por encima de ella si el salto (por encima de una sola ficha) le lleva a una casilla vacía.
c) Son válidos tanto los movimientos hacia atrás como hacia adelante.
¿Cuál es el mínimo número de movimientos necesarios para resolverlo?.
Si jugamos con n fichas de cada color, dejando una casilla vacía, ¿cuál será ahora ese número mínimo de movimientos?.
¿Y si jugamos con n fichas de cada color, pero dejando m casillas vacías en el centro?.
¿Puedes demostrar los resultados obtenidos? \cite{Juegos:11}

\chapter*{Conclusion}