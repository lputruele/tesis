% -*- Mode:TeX -*-

%% IMPORTANT: The official thesis specifications are available at:
%%            http://libraries.mit.edu/archives/thesis-specs/
%%
%%            Please verify your thesis' formatting and copyright
%%            assignment before submission.  If you notice any
%%            discrepancies between these templates and the 
%%            MIT Libraries' specs, please let us know
%%            by e-mailing thesis@mit.edu

%% The documentclass options along with the pagestyle can be used to generate
%% a technical report, a draft copy, or a regular thesis.  You may need to
%% re-specify the pagestyle after you \include  cover.tex.  For more
%% information, see the first few lines of mitthesis.cls. 

%\documentclass[12pt,vi,twoside]{mitthesis}
%%
%%  If you want your thesis copyright to you instead of MIT, use the
%%  ``vi'' option, as above.
%%
%\documentclass[12pt,twoside,leftblank]{mitthesis}
%%
%% If you want blank pages before new chapters to be labelled ``This
%% Page Intentionally Left Blank'', use the ``leftblank'' option, as
%% above. 

\documentclass[oneside,letterpaper,12pt,spanish]{report}
\usepackage{lgrind}
%% These have been added at the request of the MIT Libraries, because
%% some PDF conversions mess up the ligatures.  -LB, 1/22/2014
\usepackage{cmap}
\usepackage[T1]{fontenc}
\usepackage[utf8]{inputenc} %%%Para poner acentos directamente
\usepackage{latexsym}
\usepackage{lmodern}
%\usepackage[spanish]{babel} % division de silabas en español.
%\usepackage{ucs}
%\usepackage[utf8]{inputenc} %%%Para poner acentos directamente

%\usepackage[T1]{fontenc}
\usepackage{fancyhdr}
%\usepackage{babel}
\usepackage{makeidx}
\usepackage{graphics}  
\usepackage{algorithm}
%\usepackage{algorithmic}
\usepackage[dvips]{graphicx}
\usepackage{latexsym}
\usepackage{amssymb} 
\usepackage{amsthm}
\usepackage{url}
\usepackage{color}


\addtolength{\textwidth}{1cm} % Ancho del texto

% traducción de algunos nombres del paquete algorithm
%\floatname{algorithm}{Algoritmo}
%\renewcommand{\listalgorithmname}{Indice de algoritmos}

% Definiciones
\newtheorem{definition}{Definición}[section]

%Ejemplos
\newtheorem{ejemplo}{\normalfont \rule{0.2in}{0.11in}  {\rm Ejemplo }} [section]

%Casos de las reglas de reconstraccion
\newtheorem{caso}{\large {\rm Caso}}
 

%SUBCasos de las reglas de reconstraccion
\newtheorem{subcaso}{\large {\rm Caso}} [caso]

     
% Teoremas
\newtheorem{theorem}{Teorema}[section]

% Lemmas
\newtheorem{lema}{Lema}[section]

%%%%%%%%%%%%%%%% DEFINICIONES DE ALGUNOS COMANDOS Y AMBIENTES %%%%%%%%%%%%%%%

\newcounter{ruleAGcounter}
\newenvironment{AG}
{
 \setcounter{ruleAGcounter}{0} \begin{center} \begin{figure}[!hb]
 \rule{\textwidth}{.5pt} \footnotesize
}
{\normalsize \rule{\textwidth}{.5pt} \end{figure} \end{center}}
 
\newcommand{\newproduction}[2]{$p_\arabic{ruleAGcounter}$:
            \emph{#1} $\rightarrow$ \emph{#2}
            \addtocounter{ruleAGcounter}{1}\\}

%Por ahora al pedo esta lo que sigue
\newenvironment{micaso}[1][2] {\begin{caso} \textnormal{#1}} {\end{caso}}

%\newenvironment{proof}[1][Prueba]{}
 
\newcommand{\attribution}[1]{\hspace*{2cm}\emph{#1} \\} 

%Letra de los Estados
\newcommand{\letraEstado} [1] {\texttt{#1}}
%Letra de lso Diagramas de Clases
\newcommand{\letraDC} [1] {\texttt{#1}}
%Letra de las Funcionalidades
\newcommand{\letraFunc} [1] {\normalsize{#1}}
%Letra  del texto OR y AND
\newcommand{\letraORAND} [1] {\texttt{#1}}
%Color del texto
\newcommand{\textocolor} [1] {\textcolor{black}{#1}}
%Color del texto 2
\newcommand{\textocolorDos} [1] {\textcolor{black}{#1}}
%Color del texto 3
\newcommand{\textocolorTres} [1] {\textcolor{black}{#1}}


%letras de los elementos Ecore
\newcommand{\letraEcore} [1] {\texttt{#1}}
\newcommand{\letraRelaciones} [1] {\textbf{#1}}

\makeatother
\makeindex
 
%============================================================================

\pagestyle{plain}

%% This bit allows you to either specify only the files which you wish to
%% process, or `all' to process all files which you \include.
%% Krishna Sethuraman (1990).

\typein [\files]{Enter file names to process, (chap1,chap2 ...), or `all' to
process all files:}
\def\all{all}
\ifx\files\all \typeout{Including all files.} \else \typeout{Including only \files.} \includeonly{\files} \fi

\begin{document}


\title{
UNIVERSIDAD NACIONAL DE RIO CUARTO\\
$\;$\\
$\;$\\
\small{Trabajo de tesis para la obtenci\'on del grado de Licenciatura en Ciencias de la Computaci\'on}
$\;$\\$\;$\\
\Large{\textbf{ T\'itulo : Verificación de modelos con Cálculo-$\mu$}}
} 

\author{
                        Autor\\
                        Luciano Putruele\\
                        \\
                 Director de Tesis: \emph{Dr. Pablo F. Castro}\\ 
                 Co-Director de Tesis: \emph{Dr. Germán Regis}\\ 
}

\date{Rio Cuarto, Argentina\\ Abril 2016}

 
\maketitle
 

%Para dejar una página en blanco
\newpage
\mbox{}
\thispagestyle{empty} % para que no se numere esta página

%Incluye el abstact y los Agradecimientos.

% $Log: abstract.tex,v $
% Revision 1.1  93/05/14  14:56:25  starflt
% Initial revision
% 
% Revision 1.1  90/05/04  10:41:01  lwvanels
% Initial revision
% 
%
%% The text of your abstract and nothing else (other than comments) goes here.
%% It will be single-spaced and the rest of the text that is supposed to go on
%% the abstract page will be generated by the abstractpage environment.  This
%% file should be \input (not \include 'd) from cover.tex.
\chapter*{Resumen}
\pagenumbering{roman}
\addcontentsline{toc}{chapter}{Resumen} % si queremos que aparezca en el índice

En esta tesis desarrollaremos una herramienta de verificación de modelos (\emph Model \emph Checking), llamada MC2 (Mu Cálculus Model Checker, o MCMC, de ahi MC2), sobre modelos de sistemas formalizados en un lenguaje de descripción simple que también será desarrollado en este trabajo. La herramienta de verificación se encargará de, valga la redundancia, verificar propiedades, caracterizadas en la forma de la lógica temporal Cálculo-$\mu$ (o cálculo-Mu), sobre algún modelo descripto. Cabe destacar que tanto el lenguaje de descripción de modelos como el verificador forman parte de la misma herramienta, incluso las propiedades a verificar se deben especificar en la descripción.

Adicionalmente, los descripciones MC2 se basan en la definición de reglas de transición usando proposiciones lógicas atómicas con lo cual es transparente ver la estructura del modelo como una máquina de transición de estados más allá de que internamente se los trata simbólicamente como fórmulas para mejorar el rendimiento.

A todo esto, el metalenguaje utilizado para el desarrollo de estas herramientas es Haskell. Esto también significó una motivación para el desarrollo de esta tesis ya que la utilización de un lenguaje funcional como Haskell para el desarrollo de la herramienta en su totalidad, en lugar de la utilización lenguajes imperativos u orientados a objetos, da un enfoque distinto al comportamiento de la herramienta.

Lo que se propone con este proyecto es explorar otras alternativas para especificar propiedades que un modelo deba satisfacer, siendo la alternativa en este trabajo el Cálculo Mu, en vez de las lógicas temporales más utilizadas en este tipo de herramientas, como ser LTL, CTL y CTL*, ya que esto trae la ventaja de que el Cálculo-Mu es más expresivo que los anteriormente nombrados. MC2 puede servir como herramienta de bajo nivel, en el sentido de que se puede extraer una descripción MC2 a partir de modelos de mas alto nivel, y asi mismo, propiedades en lógicas temporales de más alto nivel, para asi despues usar el verificador MC2 sobre la descripción extraida.


%\emph{Palabras clave}: .

%\thispagestyle{empty} % para que no se numere esta página
%\end{abstract}

\chapter*{Agradecimientos}
\addcontentsline{toc}{chapter}{Agradecimientos} % si queremos que aparezca en el índice
%\thispagestyle{empty} 
{\sl Agradezco a la universidad, el departamento de computación, y en especial a mi familia y mis amigos, esto no seria posible sin ellos.}



\listoffigures
\addcontentsline{toc}{chapter}{Lista de figuras} % para que aparezca en el indice de contenidos

%Las siguintes 3 lineas es para sacarle la numeracion de paginas al Indice
%\addtocontents{toc}{\protect\thispagestyle{empty}}
\tableofcontents %put toc in
%\cleardoublepage %start new page


%\addtolength{\headwidth}{1cm} % Ancho del encabezado (fancy headers)
\renewcommand{\sectionmark}[1]{}
\renewcommand{\chaptermark}[1]{\markboth{#1}{}} % capíitulo en minísculas

%En la Introduccion se resetea el contador de paginas y el estilo de fuente de los numeros.

%% This is an example first chapter.  You should put chapter/appendix that you
%% write into a separate file, and add a line \include{yourfilename} to
%% main.tex, where `yourfilename.tex' is the name of the chapter/appendix file.
%% You can process specific files by typing their names in at the 
%% \files=
%% prompt when you run the file main.tex through LaTeX.
\chapter{Introducción}
\pagenumbering{arabic}
\noindent La verificación de modelos (\emph{model checking}) es una técnica automática para verificar modelos de sistemas con una cantidad finita de estados, por ejemplo protocolos de comunicación y diseños de circuitos, entre otros. En general, la aplicación de esta técnica consta de una caracterización del modelo en términos de algún tipo de sistema de transición de estados y la especificación de la propiedad a analizar en algún tipo de lenguaje formal, comúnmente algún tipo de lógica temporal.\\
\\
\noindent Se utiliza una búsqueda eficiente para determinar automáticamente si las especificaciones son satisfechas por el grafo \cite{Clarke:5}. Esta técnica fue desarrollada originalmente en 1981 por Clarke y Emerson. Esta técnica tiene varias ventajas importantes sobre probadores de teoremas para verificación de circuitos y protocolos. La más importante es que es automática. Normalmente, el usuario provee una representación de alto nivel del modelo y una especificación de la propiedad que se desea verificar. El model checker terminará devolviendo la respuesta True indicando que el modelo satisface la especificación o dará una traza de ejecución a modo de contra-ejemplo si el modelo no satisface la propiedad. Esta es una propiedad muy importante a la hora de encontrar bugs sutiles.

\section{Objetivos}
El objetivo principal de este proyecto es proveer una herramienta de verificación de modelos puramente funcional además de explorar otras alternativas para especificar propiedades que un modelo deba satisfacer, siendo la alternativa en este trabajo el {\mucalculo}, un lenguaje altamente expresivo, lo cuál trae la ventaja de que propiedades descritas en muchos otros tipos de lógicas temporales pueden ser descritos también por el {\mucalculo}. En cuanto a la aplicación práctica de la herramienta, la misma esta planeada para verificar propiedades en lógicas donde el problema de la verificación de modelos pueda ser reducida a {\mucalculo}, por ejemplo dCTL \cite{Castro:9}, una lógica temporal deóntica usada para especificar propiedades sobre sistemas tolerantes a fallos.

\section{Estructura}
En el capítulo 2 analizaremos conceptos básicos para la comprensión de esta tesis, conceptos como la verificación de modelos, representación de estos modelos, el concepto de lógicas temporales, y en particular, el {\mucalculo}.
En el capítulo 3 veremos formas de representar los datos de una manera más concisa e introduciremos la noción de verificación simbólica de modelos.
En el capítulo 4 nos adentramos en la herramienta MC2, su lenguaje de descripción de modelos y su verificador de modelos. Se entra en detalle con la sintaxis y la semántica del lenguaje. Para terminar, se analizarán cuestiones de diseño e implementación de la herramienta. 
En el capítulo 5 aplicaremos la herramienta a algunos casos de estudio concretos a modo de ejemplificación del uso práctico de esta herramienta.

\section{Disposición de la herramienta}

La herramienta MC2 se puede encontrar en el siguiente repositorio:\\
\\
https://github.com/lputruele/MC2-Mu-Calculus-Model-Checker




%% This is an example first chapter.  You should put chapter/appendix that you
%% write into a separate file, and add a line \include{yourfilename} to
%% main.tex, where `yourfilename.tex' is the name of the chapter/appendix file.
%% You can process specific files by typing their names in at the 
%% \files=
%% prompt when you run the file main.tex through LaTeX.
\chapter{Conceptos preliminares}

La verificación de modelos o model checking es una técnica automática de verificación de propiedades sobre sistemas con una cantidad finita de estados. Es una alternativa interesante con respecto al testing o las simulaciones ya que a diferencia de estas técnicas, el model checking hace una prueba exhaustiva del sistema, es decir, analiza todas las trazas posibles de la ejecución del sistema en cuestión. Sin embargo, esto trae un problema, esto es el problema de la explosión de estados. Esto ocurre en sistemas con muchas interacciones internas, y que pueden hacer crecer exponencialmente el espacio de estados posibles del sistema, ya que la prueba es exhaustiva no se puede ignorar ningún estado posible.
En los últimos años se ha logrado un gran progreso en cómo lidiar con este problema mediante formas más compactas de representar al sistema, como por ejemplo, una representación simbólica.

El proceso del model checking consta de varias tareas:

Modelado: Lo primero es convertir el modelo de un sistema en un formalismo aceptado por la herramienta de verificación de modelos.

Especificación: Es necesario expresar de alguna forma las propiedades que necesitan ser verificadas en el modelo. En general estas propiedades se dan en alguna lógica formal, particularmente lógicas temporales ya que estas pueden expresar comportamientos futuros del sistema.

Verificación: Dado el modelo y la especificación, la tarea de verificar significa explorar exhaustivamente todos los estados posibles del sistema para llegar a la conclusión de que el mismo satisface la especificación o no, en este último caso se suele dar también una traza de error, lo cual ayuda al programador para encontrar fallas en el sistema. Sin embargo, la causa de que el modelo no haya pasado la verificación puede deberse a una especificación incorrecta.


\section{Modelado de sistemas}
En esta sección veremos cómo representar un modelo explícitamente mediante una estructura de Kripke, más tarde veremos otra forma de representación llamada simbólica que representa el modelo mediante una fórmula lógica de primer orden.

Sea $AP$ un conjunto de proposiciones atómicas, una estructura de Kripke $M$ sobre $AP$ es una cuatro-upla $M = (S, S_{0}, R, L)$ donde
\newline1. $S$ es un conjunto finito de estados.
\newline2. $S_{0} \in S$ es el conjunto de estados iniciales.
\newline3. $R \in S \times S$ es una relación de transición total, es decir para cada estado $s \in S$ existe un estado $s' \in S$ tal que $R(s,s')$ vale.
\newline4. $L \colon S \to 2^{AP}$ es una función que etiqueta a cada estado con el conjunto de proposiciones atómicas que son verdaderas en ese estado.
Un camino en la estructura $M$ desde un estado $s$ es una secuencia infinita de estados $p = s_{0}, s_{1}, s_{2}, s_{3}, ...$, tal que $s = s_{0}$ y $R(s_{i},s_{i+1})$ vale para todo $i>0$.

Sea $V = {v_{1}, v_{2}, ..., v_{n}}$ el conjunto de variables del sistema y sea $D$ el dominio, llamaremos una valuación de $V$ a una función que asocia a cada variable de $V$ un valor de $D$.

Un estado del sistema se puede representar como una valuación de las variables del sistema. Una proposición atómica de la forma $v = d$ donde $v \in V$ y $d \in D$ será verdadera en un estado $s$ si y solo si $s(v) = d$.
Dada una valuación, podemos escribir una fórmula que sea verdadera precisamente para esa valuación, por ejemplo si tenemos $V = \{x,y,z\}$ y la valuación $(x \gets True, y \gets True, z \gets False)$ entonces derivamos la fórmula $(x \land y \land !z)$. En general, una fórmula puede ser verdadera para varias valuaciones. Si adoptamos la convención de que una fórmula representa el conjunto de todas las valuaciones que la hacen verdadera, entonces podremos describir ciertos conjuntos de estados como fórmulas de primer orden.
En particular, el conjunto de los estados iniciales del sistema puede describirse como una fórmula de primer orden $S_{0}$ sobre las variables en $V$.
Una transición del sistema se puede representar como un par ordenado de valuaciones, de forma similar podemos describir conjuntos de transiciones mediante una formula para ese par, pero para poder expresar la fórmula se necesita una copia $V'$ de $V$ para hablar del \emph siguiente estado, en V' todas las variables estan primadas. Por ejemplo si tenemos una transición $(x \gets True, y \gets True, z \gets False,(x \gets True, y \gets True, z \gets True))$, podemos derivar la fórmula $(x \land y \land !z \land x' \land y' \land z')$.

Consideremos el siguiente ejemplo, tenemos $V = \{x,y,z\}$ y $D = \{True, False\}$, $S_{0} (x,y,z) = (x= True \land y = True \land z = False)$, y tenemos solo una transición: $z := x \land y$, definimos la estructura de kripke de la siguiente manera:

\[S = D \times D \times D\]
\[S_{0} = \{(1, 1, 0)\}\]
\[R = \{((1, 1, 0), (1, 1, 1)), ((1, 1, 1), (1, 1, 1))\}\]
\[L (1, 1, 0) = \{x = 1, y = 1, z = 0\},\]
\[L (1, 1, 1) = \{x = 1, y = 1, z = 1\}\]

El único camino posible en esta estructura partiendo del estado inicial es: (1, 1, 0), (1, 1, 1), (1, 1, 1), (1, 1, 1) …

\section{Especificación de propiedades}

Ahora describiremos una lógica para especificar propiedades deseadas en una estructura de Kripke u otra máquina de transición de estados. La lógica utiliza proposiciones atómicas y operadores como la disyunción y la negación para construir expresiones más complicadas que describan propiedades sobre estados.
La lógica temporal es un formalismo que permite describir secuencias de transiciones entre estados en un sistema reactivo, nos interesa saber si en algún momento se llega a un estado determinado o que nunca se llegue a un deadlock. Para esto introduce nuevos operadores especiales que permiten hablar sobre tiempo. Estos operadores pueden combinarse con los operadores lógicos conocidos.
Analizaremos a continuación una lógica temporal muy potente llamada Cálculo-$\mu$.


\section{Cálculo-$\mu$}

Sea $M = (S, T, L)$ una estructura de Kripke y sea $VAR = {Q, Q1, Q2, …}$ un conjunto de variables relacionales, donde a cada variable relacional se le puede asignar un subconjunto de S, construimos una $\mu$-fórmula como sigue:

-Si $p$ pertenece a $AP$ entonces $p$ es una fórmula.
-Si $Q$ pertenece a $VAR$, $Q$ es una fórmula.
-Si $f$ y $g$ son fórmulas, entonces $\neg f$, $f \lor g$, y $f \land g$ son fórmulas.
-Si $f$ es una fórmula, entonces $\Box f$ y $\Diamond f$ son fórmulas.
-Si $Q \in VAR$ y $f$ es una fórmula entonces $\mu Q.f$ y $\nu Q.f$ son fórmulas

Las variables pueden estar libres o ligadas en una fórmula a través de un operador de punto fijo. Una fórmula cerrada es una fórmula sin variables libres.

El significado intuitivo de $\Diamond f$ es “Es posible realizar una transición a un estado donde f vale“, similarmente $\Box f$ significa “f vale en todos los estados alcanzables por medio de una transición“
Los operadores $\mu$ y $\nu$ expresan puntos fijos menores y mayores respectivamente. El conjunto vacío de estados se denota con $False$ y el conjunto de todos los estados $S$ se denota con $True$.

Ejemplos

-$\nu Z \cdot f \land \Box Z$ se interpreta como “f es verdadera siempre en todo camino”
-$\mu Z \cdot f \lor \Diamond Z$ se interpreta como “existe un camino hacia un estado donde f vale”
-$\nu Z \cdot \Diamond T \land \Box Z$ se interpreta como “no hay estados que no tengan transiciones hacia otros estados”

Formalmente, una fórmula $f$ se interpreta como un conjunto de estados donde $f$ es verdadera, escribimos este conjunto como $[[f]]$ sobre un sistema de transición de estados $M$ y un ambiente $e: VAR \to 2^{S}$, denotaremos $e[Q \gets W]$ como un ambiente que es igual a $e$ solo que $Q$ ahora tiene el valor $W$. el conjunto $[[f]]$ sobre $M$ y $e$ se define recursivamente de la siguiente manera:

\[ [[p]] M e = \{s \mid p \in a L(s)\} \] 
\[ [[Q]] M e = e(Q) \]
\[ [[\neg f]] M e = S \setminus [[f]] M e \]
\[ [[f \land g]] M e = [[f]] M e \cap [[g]] M e \]
\[ [[f \lor g]] M e = [[f]] M e \cup [[g]] M e \]
\[ [[\Diamond f]] M e = \{s \mid \exists t : s \to t \land t \in [[f]] M e\} \]
\[ [[\Box f]] M e = \{s \mid \forall t : s \to t  \rightarrow t \in [[f]] M e\} \]
$[[[\mu Q f]] M e$ es el menor punto fijo del predicado transformador $t:2^{S} \to 2^{S}$ definido como $t(W) = [[f]] M e[Q \gets W] $
$ [[\nu Q f]] M e$ es el mayor punto fijo del predicado transformador $t:2^{S} \to 2^{S}$ definido como $t(W) = [[f]] M e[Q \gets W] $

%\include{Bibliography}

\bibliography{tesis}  
\bibliographystyle{IEEEtran}


\chapter*{Apéndices}
\label{Apendices}
\markboth{Apendices}{} % para que cambie el encabezado, si no, usaría el del último chapter{}
\addcontentsline{toc}{chapter}{Apéndices} % para que se añada en el indice

\appendix
\chapter{Especificación de la Transformación en Query/View/Transformation Relations}
\label{Especificacion de la Transformacion en QVT/Relations QVT}


%\section{Especificación de la Transformación en QVT/ Relations QVT}
%\label{Especificación de la Transformación en QVT/Relations QVT}
Código completo, en QVT Relation, de la transformación de máquinas de estados con elementos opcionales a máquinas de estados concretas.

\begin{verbatim}
-- Inicio de la Transformación 

\end{verbatim}


%Poner el codigo de la transformación por ejemplo

\printindex
\end{document}

